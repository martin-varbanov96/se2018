\documentclass{article}
\usepackage[bulgarian]{babel}
\usepackage{tikz}
\usepackage{amsfonts}


\begin{document}






\title{Въпрос 19: Хипотези и Нейман-Пирсън}
\author{Martin Varbanov}
\date{Юни 2019}
\maketitle

\section{Дефиниции}

\subsection{Хипотези}

ще въведем няколко означения \\

$X_n$ - извадково пространство \\

$x=(x_1,..,x_n)$ - стойности на наблюденията \\

$\theta$ - параметър

$H_0 : \theta = \theta_0, \theta \in \mathbb{R}$ - Основна/нулева хипотеза \\

$H_1 : \theta = \theta_1, \theta \in \mathbb{R}$ - Алтернативна хипотеза \\



\subsubsection{Функция на правдоподобие}
$$L(x_1,..,x_n) = \prod_{i=1}^n f(x_i | \theta)$$


\subsubsection{проста хипотеза}
Когато на $H_0$ съответства точно едно разпределение $L_0(x)$ се говори за проста основна хипотеза

\subsubsection{сложна хипотеза}
когато $H_0$ не се удовлетворява от едно единствено разпределение говорим за сложна основна хипотеза




\subsection{Грешки от 1ви и 2ри род}

\begin{center}
    \begin{tabular}{| l | l | l |}
    \hline
     &  Нямаме основание да отхъвлим $H_o$ & Отхвърляме $H_0$ \\ \hline
    $H_0$ e истина & OK & Грешка от 1ви род \\ \hline
    $H_1$ е истина & Грешка от 2ри род & ОК \\ \hline
    \end{tabular}
\end{center}


\subsection{Критична област}
Тук ще дадем дефиниция за критична област:

$\overline{w} \subset X_n$ - мн-во от $(x_1,..,x_n)$, т.че $H_0$ да е вярно \\

$w = X_n \ \overline{w}$ - критична област \\



ВЪПРОС: ОСТАНА ДА НАПИШЕМ ЗА P-VALUE!!!

\subsubsection{вероятност за грешка от 1ви род/Ниво на доверие}
$$\alpha = P(x \in w|H_{0})=\int_{w} L_{0}(x)dx$$

\subsubsection{бетата}
$$\beta = P(x \notin w | H_{1})$$

\subsubsection{Мощност}
$$\Pi=P(x \in w | H_{1})$$

\subsubsection{Оптимална критична облас}

$w^{*}$ e оптимална критична област ако са изпълнени:
$$P(x \in w | H_{0}) \leq P(x \in w^{*} | H_{0})) = \alpha,\forall w$$

$$P(x \in w^{*} | H_{1}) \geq P(x \in w | H_{1})$$

\section{Лема на Неймън-Пирсън}

\subsection{Дефиницийки}
Нека $w^{*}$ e оптимална критична област и: \\

$$(\int ... \int)_{w^{*} \subset X_n} L_1(x_{1},...,x_n) dx_{1},..,dx_{n}=P(x \in w^{*} | H_{1})$$

and

$$(\int ... \int)_{w^{*} \subset X_n} L_{0}(x_1,..,x_n) dx_{1},...,dx_{n} \leq \alpha$$

\section{Лема на Неймън-Пирсън}
$M_{\alpha}$ - множество от подмножества $w \in X_{n}$, тъй че:
$$P(x \in w | H_{0}) = \int_{w} L_0(x)dx=\alpha$$

$$P(x \in w^{*} | H_{1}) = \int_{w^{*}} L_{1}(x) dx \geq \int_{w^{*}} L_{1}(x) dx, \forall w^{*} \in M_{\alpha}$$

\subsection{Лема}

При проверка на хипотеза:
$$H_{0}: L(x)=L_0(x)$$
$$H_{1}: L(x)=L_1(x)$$

с ниво на съгласие $\alpha$ и $w^{*} \in M_{\alpha}$, така че да същестува константа $K=K_{\alpha}>0$, т. че да е изпълнено:
$$L_{1} \geq K L_{0} , \forall x \in w^{*}$$
$$L_{1} \leq K L_{0}, \forall x \notin w^{*}$$
=> $w^{*}$ е оптимална крит. обл.


\subsection{Доказателство:}
Нека $w \in M_{\alpha}$ и $C=w \cap w^{*}$, a=$w^{*} \textbackslash c$, $b=w\textbackslash c$


\begin{tikzpicture}
 \begin{scope}[blend group=soft light]
    \fill[red!30!white]   (10:1.2) circle (2);
    \fill[green!30!white] (150:1.2) circle (2);
 \end{scope}
  \node at ( 10:2)    {a};
  \node at ( -130:2)  {$w$};
  \node at (150:2)    {b};
  \node at (300:2)    {$w^{*}$};
  \node [font=\Large] {c};
\end{tikzpicture}


От $w,w^{*} \in M_{\alpha}:$
$$\alpha = \int_{w} L_{0}(x) dx = \int_{b} L_{0}(x) dx + \int_{c} L_{0}(x) dx$$
$$\alpha = \int_{w^{*}} L_{0}(x) dx = \int_{a} L_{0}(x) dx + \int_{c} L_{0}(x) dx$$
от това следва:
$$\int_{b} L_{0}(x) dx = \int_{a} L_{0}(x) dx$$


Нека сега разгледаме:
$$\Delta = \int_{w^{*}} L_{1}(x) dx - \int_{w} L_{1}(x) dx=\int_{a} L_{1}(x) dx - \int_{b} L_{1}(x) dx$$

Идеята по- долу дали следва от: \\
$$L_{1}(x) \geq K L_{0} x, x \in w^{*} => \int_{w^{*}} L_{1}(x) dx \geq K \int_{a} L_{0}(x)dx $$
$$L_{1}(x) \leq K L_{0} x,x \notin w^{*} => \int_{w} L_{1}(x) dx \leq K \int_{b} L_{0}(x)dx $$
и ако умножим второ по -1 и съберем с 1вото се получава: \\


$$\Delta = \int_{w^{*}} L_{1}(x) dx - \int_{w} L_{1}(x) dx \geq K(\int_{a} L_{1}(x) dx - \int_{b} L_{1}(x) dx )$$
$$=> \Delta \geq 0$$

от тука се вижда, че условието за оптимална критична област:
$$P(x \in w^{*}|H_{1}) = \int_{w^{*}} L_{1}(x) dx \geq \int_{w} L_{0} (x) dx$$
Следва, че $w^{*}$ e оптимална критична област.

\subsection{Следствия}

$$\frac{L_{1}(x)}{L_{0}(x)} \geq K$$


$$P(\frac{L_{1}(x)}{L_{0}(x)} \geq K| H_{0})=\alpha $$


$$\pi=P(x \in w^{*} | H_{1}) = \int_{w^{*}} L_{1}(x) dx$$






\end{document}
